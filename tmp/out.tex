
%%% CV template by Moacir P. de Sá Pereira as part of the Simple-CV project:
%%%
%%% http://plain-plain-text.org/projects/simple-cv/

\documentclass[%
    11pt,
  oneside
  ]{memoir}

\usepackage{enumitem}
%%% Font declarations.
%%% please see Fonts-README.md in this repo for more information.
\usepackage{fontspec} % Choose fonts intelligently
  %%% Fontspec looks for the font set in data/pdf-options.yml.
  %%% If that font does not exist, it will complain and default to Latin Modern.
  %%%
  %%% Fonts are fiddly, so the pdf-smallcaps-headings and pdf-bold-as-smallcaps
  %%% settings in data/pdf-options.yml may not work as expected.
  %%%
  %%% Included below are some advanced OpenType features that work well with
  %%% EB Garamond. You can download EB Garamond here:
  %%%
  %%% http://www.georgduffner.at/ebgaramond/download.html
  %%%
  %%% For more information on the OpenType features available for declaration,
  %%% please see part IV of the Fontspec documentation, available here:
  %%%
  %%% http://mirrors.ctan.org/macros/latex/contrib/fontspec/fontspec.pdf
    \IfFontExistsTF{Times New Roman}{%
    \setmainfont{Times New Roman}[%
              BoldFont = Times New Roman ,
        BoldFeatures = {Letters=SmallCaps} ,
            %%% These features may not work with all fonts.
      % Ligatures=Historic ,
      % Ligatures=Rare ,
      % ItalicFeatures={Style=Swash},
      Mapping=tex-text
    ]
  }{}
  
%%% PDF generation.
\usepackage[ breaklinks=true, hidelinks ]{hyperref}
\hypersetup{%
            pdftitle={Βιογραφικό Σημείωμα της Βαρβάρας Γαρνέλη},
            pdfauthor={, Βαρβάρα Γαρνέλη.},
            pdfborder={0 0 0},
            breaklinks=true}

%%% Layout of the page
  \setlrmarginsandblock{.5in}{*}{*}
  \setulmarginsandblock{.5in}{*}{*}
  \setheadfoot{0pt}{\baselineskip} %%% footer is a baseline tall.
  \setheaderspaces{*}{0pt}{*}

%%% Define the (default) chapter style and choose the chapter
  \makechapterstyle{line}{%
    \setlength{\beforechapskip}{0pt}
    \setlength{\afterchapskip}{0pt}
    \renewcommand*{\chaptitlefont}{\Large\scshape}
    \renewcommand*{\printchaptertitle}[1]{%
    \chaptitlefont ##1 \smallskip\hrule\vspace{\baselineskip}
    }
  }
\chapterstyle{line}

%%% Section styling. We assume two, margin and overlapping.
\setsecnumdepth{subsubsection} %%% Keep the section counters running (see below).
%%% This gets complicated, because we need the numbers to roll over so that the
%%% subsections have an appropriate beforeskip.
  %%% using regular headings mode
      \setsecheadstyle{\large\scshape}
      \setaftersecskip{\baselineskip} %%% add a blank line after the heading
    \setlength{\parindent}{3em} %%% indent paragraphs
%%% Blank out all printed counters and their teeny spaces. Vital both for
%%% sections and subsections.
\makeatletter %%% enter into macro mode
\renewcommand{\@seccntformat}[1]{}
\makeatother %%% exit macro mode.

%%% Subsection styling. 
%%% Should someone (me) include subsections in their CV, things get complex.

\setsubsecheadstyle{\normalsize} %%% Subheads shouldn’t be big.
\setaftersubsecskip{0em} %%% Subheads should have no space after them
%%% New problem: \subsection should have no top margin, so it’s in line w/ the 
%%% section header. But if that’s the case, when a subsection appears later in
%%% the section, it looks just like a regular line. We need subsection to have
%%% conditional beforeskips, based on whether it’s the first subsection or not.
\let\oldsubsection\subsection %%% to avoid recursion
\renewcommand{\subsection}[1]{%
  \ifnumequal{\value{subsection}}{0} %%% If this is the first subsection
  {\setbeforesubsecskip{0em} \oldsubsection{#1}} %%% make beforeskip 0
  {\setbeforesubsecskip{\medskipamount} \oldsubsection{#1}} %%% else add a skip
}
  \setsubsecindent{3em} %%% In overlapped, yes.


%%% List styling. 
%%% They behave differently based on margin or regular headings.
\setlist[itemize]{nosep} %%% No gaps between list items.
  \setlist[itemize,1]{leftmargin=!,labelwidth=1em,labelindent=3em} %%% Push list in.
%%% And now the markers.
\renewcommand{\labelitemi}{•}
\renewcommand{\labelitemii}{-}
\renewcommand{\labelitemiii}{*}

\checkandfixthelayout

%%% Format footer
\copypagestyle{chapter}{plain}
  \makeoddhead{chapter}{}{}{}
  \makeoddfoot{chapter}{}{}{%
    , Βαρβάρα Γαρνέλη.,  \today, \thepage}
\pagestyle{chapter}

%%% Redefine \section to tighten vertical space
\let\oldsection\section
\renewcommand{\section}[1]{%
  \oldsection{#1}
  \leavevmode
  \par
  \vspace{\dimexpr-\baselineskip-\parskip}
}


\begin{document}

      \chapter*{Βαρβάρα Γαρνέλη}
  

  \hypertarget{contact-information}{%
  \section{Contact Information}\label{contact-information}}
    \begin{minipage}[t]{0.3\textwidth}
      
    \end{minipage}
    \begin{minipage}[t]{0.7\textwidth}
                        {\textit{E-mail:}} c13@garn.gr \\
                                        {\textit{GitHub:}} \href{http://github.com/bgarnb}{@bgarnb}
            \end{minipage}
  \hypertarget{ux3b5ux3beux3b5ux3b9ux3b4ux3b9ux3baux3b5ux3c5ux3c3ux3b7}{%
\section{Εξειδικευση:}\label{ux3b5ux3beux3b5ux3b9ux3b4ux3b9ux3baux3b5ux3c5ux3c3ux3b7}}

Computing Education, Computational Thinking, Video Games, Computer
Programming

\hypertarget{ux3b5ux3baux3c0ux3b1ux3afux3b4ux3b5ux3c5ux3c3ux3b7}{%
\section{Εκπαίδευση}\label{ux3b5ux3baux3c0ux3b1ux3afux3b4ux3b5ux3c5ux3c3ux3b7}}

\begin{itemize}
\tightlist
\item
  \textbf{Διδακτορικό, Ιόνιο Πανεπιστήμιο, Τμήμα Πληροφορικής}, 2017

  \begin{itemize}
  \tightlist
  \item
    Video-game making approach in science education: exploring
    computational thinking skills development and student motivation
  \end{itemize}
\item
  \textbf{Μεταπτυχιακό Δίπλωμα στην Πληροφορική και τις Ανθρωπιστικές
  Σπουδές, Ιόνιο Πανεπιστήμιο, Τμήμα Πληροφορικής}, 2013
\item
  \textbf{Πτυχίο Παιδαγωγικών Σπουδών, ΣΕΛΕΤΕ, Τμήμα Πτυχιούχων ΑΕΙ},
  1993
\item
  \textbf{Πτυχίο Πληροφορικής, Οικονομικό Πανεπιστήμιο Αθήνας, Τμήμα
  Πληροφορικής}, 1992
\end{itemize}

\hypertarget{ux3b4ux3b7ux3bcux3bfux3c3ux3b9ux3b5ux3cdux3c3ux3b5ux3b9ux3c2}{%
\section{Δημοσιεύσεις}\label{ux3b4ux3b7ux3bcux3bfux3c3ux3b9ux3b5ux3cdux3c3ux3b5ux3b9ux3c2}}

\begin{itemize}
\tightlist
\item
  \textbf{Σε Διεθνή Περιοδικά:}
\end{itemize}

Garneli, V., Giannakos, M., \& Chorianopoulos, K. (2017). ``Serious
games as a malleable learning medium: The effects of narrative,
gameplay, and making on students' performance and attitudes'',
\emph{British Journal of Educational Technology}, 48(3), 842-859.

Garneli, V., \& Chorianopoulos, K. (2018). ``Programming video games and
simulations in science education: exploring computational thinking
through code analysis'', \emph{Interactive Learning Environments},
26(3), 386-401.

Garneli, V., \& Chorianopoulos, K. (2019). ``The effects of video game
making within science content on student computational thinking skills
and performance'' \emph{Interactive Technology and Smart Education}.

Garneli, V., Patiniotis, K., \& Chorianopoulos, K. (2019). ``Integrating
Science Tasks and Puzzles in Computer Role Playing Games''.
\emph{Multimodal Technologies and Interaction}, 3(3), 55.

\begin{itemize}
\tightlist
\item
  \textbf{Σε Διεθνή Συνέδρια:}
\end{itemize}

Garneli, B., Giannakos, M. N., Chorianopoulos, K., \& Jaccheri, L.
(2013, September). ``Learning by playing and learning by making''. In
\emph{International Conference on Serious Games Development and
Applications} (pp.~76-85). Springer, Berlin, Heidelberg.

Garneli, V. (2014, July). ``Instructional media and teaching methods for
engaging children with computer programming''. In 2014 \emph{IEEE 14th
International Conference on Advanced Learning Technologies}
(pp.~768-770). IEEE.

Garneli, V., Giannakos, M. N., \& Chorianopoulos, K. (2015, March).
``Computing education in K-12 schools: A review of the literature''. In
2015 \emph{IEEE Global Engineering Education Conference (EDUCON)}
(pp.~543-551). IEEE.

Garneli, V., Giannakos, M. N., Chorianopoulos, K., \& Jaccheri, L.
(2015, May). ``Serious game development as a creative learning
experience: lessons learnt''. In 2015 \emph{IEEE/ACM 4th International
Workshop on Games and Software Engineering} (pp.~36-42). IEEE.

Vassilakis, N., Garneli, V., Patiniotis, K., Deliyannis, I., \&
Chorianopoulos, K. (2019, September). "Adapting a Classic Platform Video
Game to the Carbohydrate Counting Method for Insulin-Dependent
Diabetics. In Proceedings of the 5th \emph{EAI International Conference
on Smart Objects and Technologies for Social Good} (pp.~149-154). ACM.

\hypertarget{ux3c0ux3c1ux3bfux3cbux3c0ux3b7ux3c1ux3b5ux3c3ux3afux3b1}{%
\section{Προϋπηρεσία}\label{ux3c0ux3c1ux3bfux3cbux3c0ux3b7ux3c1ux3b5ux3c3ux3afux3b1}}

\begin{itemize}
\tightlist
\item
  \textbf{Τμήμα Πληροφορικής, Ιόνιο Πανεπιστήμιο}

  \begin{itemize}
  \tightlist
  \item
    Κινητά και Κοινωνικά Μέσα, 2018
  \item
    Interaction Design, με έμφαση στη σχεδίαση και την κατασκευή
    εκπαιδευτικού λογισμικού, 2017
  \end{itemize}
\item
  \textbf{Δευτεροβάθμια Εκπαίδευση}

  \begin{itemize}
  \tightlist
  \item
    Πληροφορική, 1-1993 μέχρι σήμερα
  \end{itemize}
\item
  \textbf{Υπάλληλος γραφείου -- προγραμματιστής σε ναυτιλιακή εταιρεία},
  12/1991 -- 11/1992
\end{itemize}

\hypertarget{ux3beux3adux3bdux3b5ux3c2-ux3b3ux3bbux3ceux3c3ux3c3ux3b5ux3c2}{%
\section{Ξένες
Γλώσσες}\label{ux3beux3adux3bdux3b5ux3c2-ux3b3ux3bbux3ceux3c3ux3c3ux3b5ux3c2}}

\begin{itemize}
\tightlist
\item
  English.
\end{itemize}

\end{document}
